\documentclass[man]{apa6}
\usepackage{lmodern}
\usepackage{amssymb,amsmath}
\usepackage{ifxetex,ifluatex}
\usepackage{fixltx2e} % provides \textsubscript
\ifnum 0\ifxetex 1\fi\ifluatex 1\fi=0 % if pdftex
  \usepackage[T1]{fontenc}
  \usepackage[utf8]{inputenc}
\else % if luatex or xelatex
  \ifxetex
    \usepackage{mathspec}
  \else
    \usepackage{fontspec}
  \fi
  \defaultfontfeatures{Ligatures=TeX,Scale=MatchLowercase}
\fi
% use upquote if available, for straight quotes in verbatim environments
\IfFileExists{upquote.sty}{\usepackage{upquote}}{}
% use microtype if available
\IfFileExists{microtype.sty}{%
\usepackage[]{microtype}
\UseMicrotypeSet[protrusion]{basicmath} % disable protrusion for tt fonts
}{}
\PassOptionsToPackage{hyphens}{url} % url is loaded by hyperref
\usepackage[unicode=true]{hyperref}
\hypersetup{
            pdftitle={Data analysis in RStudio: Comparative communication. Study 1: Initial appraisal of implicit and explicit differences. Extension: The role of group membership},
            pdfauthor={Yujing Liang},
            pdfkeywords={keywords},
            pdfborder={0 0 0},
            breaklinks=true}
\urlstyle{same}  % don't use monospace font for urls
\usepackage{graphicx,grffile}
\makeatletter
\def\maxwidth{\ifdim\Gin@nat@width>\linewidth\linewidth\else\Gin@nat@width\fi}
\def\maxheight{\ifdim\Gin@nat@height>\textheight\textheight\else\Gin@nat@height\fi}
\makeatother
% Scale images if necessary, so that they will not overflow the page
% margins by default, and it is still possible to overwrite the defaults
% using explicit options in \includegraphics[width, height, ...]{}
\setkeys{Gin}{width=\maxwidth,height=\maxheight,keepaspectratio}
\IfFileExists{parskip.sty}{%
\usepackage{parskip}
}{% else
\setlength{\parindent}{0pt}
\setlength{\parskip}{6pt plus 2pt minus 1pt}
}
\setlength{\emergencystretch}{3em}  % prevent overfull lines
\providecommand{\tightlist}{%
  \setlength{\itemsep}{0pt}\setlength{\parskip}{0pt}}
\setcounter{secnumdepth}{0}
% Redefines (sub)paragraphs to behave more like sections
\ifx\paragraph\undefined\else
\let\oldparagraph\paragraph
\renewcommand{\paragraph}[1]{\oldparagraph{#1}\mbox{}}
\fi
\ifx\subparagraph\undefined\else
\let\oldsubparagraph\subparagraph
\renewcommand{\subparagraph}[1]{\oldsubparagraph{#1}\mbox{}}
\fi

% set default figure placement to htbp
\makeatletter
\def\fps@figure{htbp}
\makeatother

\shorttitle{Statistics VI Assigment}
\affiliation{
\vspace{0.5cm}
\textsuperscript{1} KU Leuven}
\keywords{keywords\newline\indent Word count: X}
\usepackage{csquotes}
\usepackage{upgreek}
\captionsetup{font=singlespacing,justification=justified}

\usepackage{longtable}
\usepackage{lscape}
\usepackage{multirow}
\usepackage{tabularx}
\usepackage[flushleft]{threeparttable}
\usepackage{threeparttablex}

\newenvironment{lltable}{\begin{landscape}\begin{center}\begin{ThreePartTable}}{\end{ThreePartTable}\end{center}\end{landscape}}

\makeatletter
\newcommand\LastLTentrywidth{1em}
\newlength\longtablewidth
\setlength{\longtablewidth}{1in}
\newcommand{\getlongtablewidth}{\begingroup \ifcsname LT@\roman{LT@tables}\endcsname \global\longtablewidth=0pt \renewcommand{\LT@entry}[2]{\global\advance\longtablewidth by ##2\relax\gdef\LastLTentrywidth{##2}}\@nameuse{LT@\roman{LT@tables}} \fi \endgroup}


\DeclareDelayedFloatFlavor{ThreePartTable}{table}
\DeclareDelayedFloatFlavor{lltable}{table}
\DeclareDelayedFloatFlavor*{longtable}{table}
\makeatletter
\renewcommand{\efloat@iwrite}[1]{\immediate\expandafter\protected@write\csname efloat@post#1\endcsname{}}
\makeatother

\title{Data analysis in RStudio: Comparative communication. Study 1: Initial
appraisal of implicit and explicit differences. Extension: The role of
group membership}
\author{Yujing Liang\textsuperscript{1}}
\date{}

\authornote{

Correspondence concerning this article should be addressed to Yujing
Liang, Tiensestraat 102, 3000 Leuven. E-mail:
\href{mailto:yujing.liang@student.kuleuven.com}{\nolinkurl{yujing.liang@student.kuleuven.com}}}

\abstract{

}

\begin{document}
\maketitle

\begin{figure}
\centering
\includegraphics{intern_rcode_files/figure-latex/unnamed-chunk-1-1.pdf}
\caption{\label{fig:unnamed-chunk-1}Interaction effect of valence and
membership on the truth of gender-related claims.}
\end{figure}

\begin{figure}
\centering
\includegraphics{intern_rcode_files/figure-latex/unnamed-chunk-2-1.pdf}
\caption{\label{fig:unnamed-chunk-2}Interaction effect of valence and
membership on the acceptability of gender-related claims.}
\end{figure}

\begin{figure}
\centering
\includegraphics{intern_rcode_files/figure-latex/unnamed-chunk-3-1.pdf}
\caption{\label{fig:unnamed-chunk-3}Interaction effect of valence and
membership on the truth of age-related claims.}
\end{figure}

\begin{figure}
\centering
\includegraphics{intern_rcode_files/figure-latex/unnamed-chunk-4-1.pdf}
\caption{\label{fig:unnamed-chunk-4}Interaction effect of valence and
membership on the acceptability of age-related claims.}
\end{figure}

\begin{figure}
\centering
\includegraphics{intern_rcode_files/figure-latex/unnamed-chunk-5-1.pdf}
\caption{\label{fig:unnamed-chunk-5}Three way interaction between
consistency, valence and membership on the familiarity of gender-related
claims.}
\end{figure}

\begin{figure}
\centering
\includegraphics{intern_rcode_files/figure-latex/unnamed-chunk-6-1.pdf}
\caption{\label{fig:unnamed-chunk-6}Three way interaction between format,
valence and membership on the familiarity of gender-related claims.}
\end{figure}

\begin{figure}
\centering
\includegraphics{intern_rcode_files/figure-latex/unnamed-chunk-7-1.pdf}
\caption{\label{fig:unnamed-chunk-7}Three way interaction between
consistency, valence and membership on the familiarity of age-related
claims.}
\end{figure}

\begin{figure}
\centering
\includegraphics{intern_rcode_files/figure-latex/unnamed-chunk-8-1.pdf}
\caption{\label{fig:unnamed-chunk-8}Three way interaction between format,
valence and membership on the familiarity of age-related claims.}
\end{figure}

\section{Data analysis}\label{data-analysis}

\textbf{RStudio Package.} I used R (Version 3.5.1; R Core Team, 2018)
and the R-packages \emph{car} (Version 3.0.6; Fox \& Weisberg, 2019;
Fox, Weisberg, \& Price, 2018), \emph{carData} (Version 3.0.2; Fox et
al., 2018), \emph{dplyr} (Version 0.8.3; Wickham,
Fran\textless{}U+00E7\textgreater{}ois, Henry, \& Müller, 2019),
\emph{emmeans} (Version 1.4.3.1; R. Lenth, 2019), \emph{fancycut}
(Version 0.1.2; Rich, 2018), \emph{ggplot2} (Version 3.2.1; Wickham,
2016), \emph{ggpubr} (Version 0.2.3; Kassambara, 2019), \emph{haven}
(Version 2.2.0; Wickham \& Miller, 2019), \emph{lsmeans} (Version
2.30.0; R. V. Lenth, 2016), \emph{lsr} (Version 0.5; Navarro, 2015),
\emph{magrittr} (Version 1.5; Bache \& Wickham, 2014), \emph{MASS}
(Version 7.3.50; Venables \& Ripley, 2002), \emph{multcompView} (Version
0.1.7; Graves, Piepho, \& Sundar Dorai-Raj, 2015), \emph{numform}
(Version 0.5.0; Rinker, 2018), \emph{papaja} (Version 0.1.0.9842; Aust
\& Barth, 2018), \emph{psych} (Version 1.8.12; Revelle, 2018),
\emph{purrr} (Version 0.3.3; Henry \& Wickham, 2019), \emph{pwr}
(Version 1.2.2; Champely, 2018), \emph{reshape2} (Version 1.4.3;
Wickham, 2007), and \emph{sjstats} (Version 0.17.7; Lüdecke, 2019) for
all the analyses.

\textbf{Dataset.} I conducted the analysis using the data set
Study1\_ready\_short.

\textbf{Data cleansing.} I cleaned the data by removing subjects who do
not identify with any gender category (gender = 3) and the rows with
missing values (NAs).

\textbf{Subdatasets.} Before executing main analysis in R, I created
sub-datasets for each dependent variables (truth, acceptability,
familiarity, stereotypicality, and positivity), for each experiments
(experiment 1a and 1b).

Before testing the predictions for judgments of truth, I first extracted
and stacked the columns (\enquote{ID}, \enquote{gender},
\enquote{Consistency}, \enquote{Format}, \enquote{TruthMenPos},
\enquote{TruthWomenPos}, \enquote{TruthMenNeg}, \enquote{TruthWomenNeg})
to create the data set my\_data\_gender\_T. Then, I created the data set
my\_data\_age\_T, which includes the columns named: \enquote{ID},
\enquote{gender}, \enquote{Consistency}, \enquote{Format},
\enquote{TruthOldPos}, \enquote{TruthYoungPos}, \enquote{TruthOldNeg},
\enquote{TruthYoungNeg}.

In the same way, I created the following subdatasets:
my\_data\_gender\_A (acceptability), my\_data\_age\_A,
my\_data\_gender\_F (familiarity), my\_data\_age\_F, my\_data\_gender\_S
(stereotypicality), my\_data\_age\_S, my\_data\_gender\_P (positivity),
and my\_data\_age\_P.

\textbf{Gender groups.} The cleaned dataset comprised of 85 male
subjects and 98 female subjects.

\textbf{Age groups.} Our subjects comprised of 61 younger people, 78
middle-aged people and 44 older people. Among them, 2 younger subjects
and 15 older subjects identified themselves as middle-aged. None of the
subjects identified with the \enquote{wrong} age group (younger
participants identifying with older people, or older participants
identifying with younger people). Such that I used their
\enquote{subjective} age group in the analyses (young: 59, middle-aged:
95, old: 29). Moreover, I distinguish them between 3 rather than 2 age
groups: younger, middle-aged, older (middle-aged = 51.91\%
\textgreater{} 25\%).

\subsection{\texorpdfstring{\textbf{Analysis
plan.}}{Analysis plan.}}\label{analysis-plan.}

\textbf{Main effect and interaction effect on the judgement of truth.}
According to the pre-analysis plan that we registered, first, a linear
regression will be performed on data sets my\_data\_gender\_T and
my\_data\_age\_T, which involves testing the main effect of valence on
the judgments of truth. Then a two way ANOVA will be carried out to test
the interaction effect between group membership and valence and to test
2 planned contrasts of the interaction.

\textbf{Main effect and interaction effect on the judgement of social
acceptability.} Accordingly, the regression analysis on data sets
my\_data\_gender\_A and my\_data\_age\_A involve testing the main effect
of valence on the judgments of acceptability, and testing the
interaction effect between group membership and valence (1 planned
contrast).

\textbf{Exploratory analysis.} In the exploratory analysis, the
regression analysis will be performed on related subdatasets. The
analysis involves testing the main effect of group membership and the
interaction effect between group membership and valence on the perceived
familiarity, stereotypicality and positivity. Further, a linear
regression will be carried out to test if consistency and format of the
claims affect how group membership and valence affect various dependent
variables.

\section{Results of Judgments of
truth}\label{results-of-judgments-of-truth}

\textbf{Analyses for Experiment 1a (Gender-related claims).} A
significant main effect of valence (positive, negative) on the judgments
of truth (\(t [730]\) = -2.40 , \emph{p} = .017) was found, with
positively valenced claims (\emph{M} = 4.15, \emph{SD} = 1.05) receiving
higher scores on truth than negatively valenced ones (\emph{M} = 3.97,
\emph{SD} = 1.05).

Additionally no significant interaction was found between valence and
group membership (ingroup, outgroup) on the judgments of truth
(\(F [1, 728]\) = 2.74, \emph{p} = .098).

Planned contrasts showed that subjects believed positively valenced
claims were significantly truer than negative valenced ones when the
claims are targeted at their ingroup (\(t [728]\) = -2.87 , \emph{p} =
.004), but there were no difference between valences when the claims are
targeted at their outgroup (\(t [728]\) = -0.53 , \emph{p} = .598).
Moreover, no differences were found between ingroupers and outgroupers
on the the judgement of truth in the positive condition (\(t [728]\) =
-1.68 , \emph{p} = .093) , and negative condition (\(t [728]\) = 0.66 ,
\emph{p} = .511).

\textbf{Analyses for Experiment 1b (Age-related claims).} A significant
main effect of valence (positive, negative) on the judgments of truth
(\(t [730]\) = -3.80 , \emph{p} = .000) was found, with positively
valenced claims (\emph{M} = 4.09, \emph{SD} = 1.21) receiving higher
scores on truth than negatively valenced ones (\emph{M} = 3.77,
\emph{SD} = 1.08).

Additionally no significant interaction was found between valence and
group membership (ingroup, outgroup, middle-aged) on the judgments of
truth (\(F [2, 726]\) = 0.68, \emph{p} = .505).

Planned contrasts showed that subjects tend to rate positively valenced
claims as significantly truer than negative valenced ones in the ingroup
condition (\(t [726]\) = -2.53 , \emph{p} = .011) and in the outgroup
condition (\(t [726]\) = -2.38 , \emph{p} = .018). Yet among middle-aged
people, the difference is marginally significant (\(t [726]\) = -1.93 ,
\emph{p} = .054). Futher, there were no significant difference between
ingroupers and outgroupers on the the judgement of truth when the claims
are either positively valenced (\(t [726]\) = -0.35 , \emph{p} = .933),
or negatively valenced (\(t [726]\) = -0.51 , \emph{p} = .865).

\section{Results of Judgments of
acceptability}\label{results-of-judgments-of-acceptability}

\textbf{Analyses for Experiment 1a (Gender-related claims).} A
significant main effect of valence (positive, negative) on the judgments
of acceptability (\(t [730]\) = -6.38 , \emph{p} = .000) was found, with
positively valenced claims (\emph{M} = 4.51, \emph{SD} = 1.34) receiving
higher scores on acceptability than negatively valenced ones (\emph{M} =
3.88, \emph{SD} = 1.32).

Additionally no significant interaction was found between valence and
group membership (ingroup, outgroup) on the judgments of acceptability
(\(F [1, 728]\) = 2.09, \emph{p} = .149).

Planned contrasts showed that subjects are not significantly more
acceptable to positive claims that targeted at their ingroup than those
targeted at outgroup (\(t [728]\) = -0.96 , \emph{p} = .338). Also,
subjects are not significantly more acceptable to negative claims which
targeted at their outgroup than those targeted at ingroup (\(t [728]\) =
1.09 , \emph{p} = .278).

\textbf{Analyses for Experiment 1b (Age-related claims).} A significant
main effect of valence (positive, negative) on the judgments of
acceptability (\(t [730]\) = -8.41 , \emph{p} = .000) was found, with
positively valenced claims (\emph{M} = 4.67, \emph{SD} = 1.35) receiving
higher scores on acceptability than negatively valenced ones (\emph{M} =
3.85, \emph{SD} = 1.27).

Additionally no significant interaction was found between valence and
group membership (ingroup, outgroup, middle-aged) on the judgments of
acceptability (\(F [2, 726]\) = 0.70, \emph{p} = .498).

Planned contrasts showed that subjects are not significantly more
acceptable to positive claims that targeted at their ingroup than those
targeted at outgroup (\(t [726]\) = 0.21 , \emph{p} = .977). Also,
subjects are not significantly more acceptable to negative claims which
targeted at their outgroup than those targeted at ingroup (\(t [726]\) =
0.03 , \emph{p} = .999).

\section{Results of exploratory
analysis}\label{results-of-exploratory-analysis}

\subsection{\texorpdfstring{\textbf{Judgments of
familiarity}}{Judgments of familiarity}}\label{judgments-of-familiarity}

\textbf{Analyses for Experiment 1a (Gender-related claims).} There was
no significant main effect of membership (ingroup, outgroup) on the
judgments of familiarity (\(t [730]\) = -0.18 , \emph{p} = .857), with
ingroup-targeted cliams (\emph{M} = 4.09, \emph{SD} = 1.39) receiving
slightly higher scores on familiarity than those targeted at outgoup
(\emph{M} = 4.08, \emph{SD} = 1.41).

Additionally no significant interaction was found between valence
(positive, negative) and group membership on the judgments of
familiarity (\(F [1, 728]\) = 0.03, \emph{p} = .866).

The three-way interaction between consistency (stereotypical,
counter-stereotypical), valence and group membership is also not
significant (\(F [1, 724]\) = 0.07, \emph{p} = .785). The three-way
interaction between format (implicit, explicit), valence and group
membership is also not significant (\(F [1, 724]\) = 0.46, \emph{p} =
.497).

\textbf{Analyses for Experiment 1b (Age-related claims).} A one-way
ANOVA was conducted to compare the effect of group membership on
judgments of familiarity, in ingroup condition (\emph{M} = 4.09,
\emph{SD} = 1.56), outgroup condition (\emph{M} = 4.18, \emph{SD} =
1.51) and middle group condition (\emph{M} = 3.97, \emph{SD} = 1.57). No
significant effect of membership on perceived familiarity
(\(F [2, 729]\) = 1.19 , \emph{p} = .305) was found.

Additionally no significant interaction was found between valence
(positive, negative) and group membership on the judgments of
familiarity (\(F [2, 726]\) = 0.43, \emph{p} = .653).

The three-way interaction between consistency (stereotypical,
counter-stereotypical), valence and group membership is also not
significant (\(F [2, 720]\) = 1.34, \emph{p} = .262). The three-way
interaction between format (implicit, explicit), valence and group
membership is also not significant (\(F [2, 720]\) = 0.23, \emph{p} =
.793). All the Post-Hoc Contrasts are also not significant, except that
while the presented claims are explicit (format level=2), membership has
marginally significant effect on the judgement of familiarity
(\(F [2, 342]\) = 2.48, \emph{p} = .085).

\subsection{\texorpdfstring{\textbf{Judgments of
stereotypicality}}{Judgments of stereotypicality}}\label{judgments-of-stereotypicality}

\textbf{Analyses for Experiment 1a (Gender-related claims).} There was
no significant main effect of membership (ingroup, outgroup) on the
judgments of stereotypicality (\(t [730]\) = -0.54 , \emph{p} = .591),
with ingroup-targeted cliams (\emph{M} = 4.47, \emph{SD} = 1.48)
receiving slightly higher scores on stereotypicality than those targeted
at outgoup (\emph{M} = 4.41, \emph{SD} = 1.51).

Additionally no significant interaction was found between valence
(positive, negative) and group membership on the judgments of
stereotypicality (\(F [1, 728]\) = 0.03, \emph{p} = .855).

The three-way interaction between consistency (stereotypical,
counter-stereotypical), valence and group membership is also not
significant (\(F [1, 724]\) = 0.46, \emph{p} = .496). The three-way
interaction between format (implicit, explicit), valence and group
membership is also not significant (\(F [1, 724]\) = 0.14, \emph{p} =
.711).

\textbf{Analyses for Experiment 1b (Age-related claims).} A one-way
ANOVA was conducted to compare the effect of group membership on
judgments of stereotypicality, in ingroup condition (\emph{M} = 4.56,
\emph{SD} = 1.57), outgroup condition (\emph{M} = 4.59, \emph{SD} =
1.58) and middle group condition (\emph{M} = 4.01, \emph{SD} = 1.57). A
significant effect of membership on perceived stereotypicality
(\(F [2, 729]\) = 11.69 , \emph{p} = .000) was found.

Specifically, middle group members tend to judge positive cliams as
significantly less stereotypical, compared to ourgoupers(\(t [726]\) =
3.26 , \emph{p} = .003), and marginally significantly less stereotypical
compared to ingroupers (\(t [726]\) = 2.26 , \emph{p} = .062). Also,
middle groupers perceive the negatively velanced cliams as significantly
less stereotypical, compared to ingroupers (\(t [726]\) = 3.13 ,
\emph{p} = .005) and ourgoupers (\(t [726]\) = 2.45 , \emph{p} = .038).
Additionally, subjects perceive positively valenced cliams as less
stereotypical when cliams are targeted at their ingroup than those
targeted at outgroup (\(t [726]\) = -0.85 , \emph{p} = .670), however,
when the cliams are negatively velanced, ingroupers would perceive
presented claims as more sterotypical compared to outgroupers
(\(t [726]\) = 0.58 , \emph{p} = .833). Both of the differences are not
significant at the alpha level of 0.05.

Additionally no significant interaction was found between valence
(positive, negative) and group membership on the judgments of
stereotypicality (\(F [2, 726]\) = 0.51, \emph{p} = .600).

The three-way interaction between consistency (stereotypical,
counter-stereotypical), valence and group membership is not significant
(\(F [2, 720]\) = 0.86, \emph{p} = .424). The three-way interaction
between format (implicit, explicit), valence and group membership is
also not significant (\(F [2, 720]\) = 0.15, \emph{p} = .861).

\subsection{\texorpdfstring{\textbf{Judgments of
positivity}}{Judgments of positivity}}\label{judgments-of-positivity}

\textbf{Analyses for Experiment 1a (Gender-related claims).} There was
no significant main effect of membership (ingroup, outgroup) on the
judgments of positivity (\(t [730]\) = 0.13 , \emph{p} = .900), with
outgroup-targeted cliams (\emph{M} = 3.79, \emph{SD} = 1.44) receiving
slightly higher scores on positivity than those targeted at ingroup
(\emph{M} = 3.77, \emph{SD} = 1.51).

Additionally there is a marginally significant interaction was found
between valence (positive, negative) and group membership on the
judgments of positivity (\(F [1, 728]\) = 3.14, \emph{p} = .077).

Post-hoc comparisons showed subjects believed that positively valenced
claims were significantly more positive than negative valenced ones both
when the they are targeted at ingroup (\(t [728]\) = 16.71 , \emph{p} =
.000) and outgroup (\(t [728]\) = 14.20 , \emph{p} = .000). Moreover, no
significant differences were found between ingroupers and outgroupers on
the the judgement of positivity in the positive condition (\(t [728]\) =
1.14 , \emph{p} = .255) , and negative condition (\(t [728]\) = -1.37 ,
\emph{p} = .172).

The three-way interaction between consistency (stereotypical,
counter-stereotypical), valence and group membership is also not
significant (\(F [1, 724]\) = 0.06, \emph{p} = .803). The three-way
interaction between format (implicit, explicit), valence and group
membership is also not significant (\(F [1, 724]\) = 0.08, \emph{p} =
.779).

\textbf{Analyses for Experiment 1b (Age-related claims).} A one-way
between subjects ANOVA was conducted to compare the effect of group
membership on judgments of positivity, in ingroup condition (\emph{M} =
3.74, \emph{SD} = 1.46), outgroup condition (\emph{M} = 3.73, \emph{SD}
= 1.46) and middle group condition (\emph{M} = 3.70, \emph{SD} = 1.54).
No significant effect of membership on perceived positivity
(\(F [2, 729]\) = 0.07 , \emph{p} = .933) was found.

Additionally no significant interaction was found between valence
(positive, negative) and group membership on the judgments of positivity
(\(F [2, 726]\) = 0.02, \emph{p} = .980).

The three-way interaction between consistency (stereotypical,
counter-stereotypical), valence and group membership is also not
significant (\(F [2, 720]\) = 0.23, \emph{p} = .797). The three-way
interaction between format (implicit, explicit), valence and group
membership is significant (\(F [2, 720]\) = 3.38, \emph{p} = .035).
Specifically, valence (positive, negative) has a significant effect on
positivity (\(F [1, 720]\) = 655.26, \emph{p} = .000). Also, there is a
significant interaction between format and valence (\(F [1, 720]\) =
59.55, \emph{p} = .000).

\begin{figure}
\centering
\includegraphics{intern_rcode_files/figure-latex/unnamed-chunk-9-1.pdf}
\caption{\label{fig:unnamed-chunk-9}Three way interaction between
consistency, valence and membership on the stereotypicality of
gender-related claims.}
\end{figure}

\begin{figure}
\centering
\includegraphics{intern_rcode_files/figure-latex/unnamed-chunk-10-1.pdf}
\caption{\label{fig:unnamed-chunk-10}Three way interaction between format,
valence and membership on the stereotypicality of gender-related
claims.}
\end{figure}

\begin{figure}
\centering
\includegraphics{intern_rcode_files/figure-latex/unnamed-chunk-11-1.pdf}
\caption{\label{fig:unnamed-chunk-11}Three way interaction between
consistency, valence and membership on the stereotypicality of
age-related claims.}
\end{figure}

\begin{figure}
\centering
\includegraphics{intern_rcode_files/figure-latex/unnamed-chunk-12-1.pdf}
\caption{\label{fig:unnamed-chunk-12}Three way interaction between format,
valence and membership on the stereotypicality of age-related claims.}
\end{figure}

\begin{figure}
\centering
\includegraphics{intern_rcode_files/figure-latex/unnamed-chunk-13-1.pdf}
\caption{\label{fig:unnamed-chunk-13}Three way interaction between
consistency, valence and membership on the positivity of gender-related
claims.}
\end{figure}

\begin{figure}
\centering
\includegraphics{intern_rcode_files/figure-latex/unnamed-chunk-14-1.pdf}
\caption{\label{fig:unnamed-chunk-14}Three way interaction between format,
valence and membership on the positivity of gender-related claims.}
\end{figure}

\begin{figure}
\centering
\includegraphics{intern_rcode_files/figure-latex/unnamed-chunk-15-1.pdf}
\caption{\label{fig:unnamed-chunk-15}Three way interaction between
consistency, valence and membership on the positivity of age-related
claims.}
\end{figure}

\begin{figure}
\centering
\includegraphics{intern_rcode_files/figure-latex/unnamed-chunk-16-1.pdf}
\caption{\label{fig:unnamed-chunk-16}Three way interaction between format,
valence and membership on the positivity of age-related claims.}
\end{figure}

\newpage

\section{References}\label{references}

\begingroup
\setlength{\parindent}{-0.5in} \setlength{\leftskip}{0.5in}

\hypertarget{refs}{}
\hypertarget{ref-R-papaja}{}
Aust, F., \& Barth, M. (2018). \emph{papaja: Create APA manuscripts with
R Markdown}. Retrieved from \url{https://github.com/crsh/papaja}

\hypertarget{ref-R-magrittr}{}
Bache, S. M., \& Wickham, H. (2014). \emph{Magrittr: A forward-pipe
operator for r}. Retrieved from
\url{https://CRAN.R-project.org/package=magrittr}

\hypertarget{ref-R-pwr}{}
Champely, S. (2018). \emph{Pwr: Basic functions for power analysis}.
Retrieved from \url{https://CRAN.R-project.org/package=pwr}

\hypertarget{ref-R-car}{}
Fox, J., \& Weisberg, S. (2019). \emph{An R companion to applied
regression} (Third.). Thousand Oaks CA: Sage. Retrieved from
\url{https://socialsciences.mcmaster.ca/jfox/Books/Companion/}

\hypertarget{ref-R-carData}{}
Fox, J., Weisberg, S., \& Price, B. (2018). \emph{CarData: Companion to
applied regression data sets}. Retrieved from
\url{https://CRAN.R-project.org/package=carData}

\hypertarget{ref-R-multcompView}{}
Graves, S., Piepho, H.-P., \& Sundar Dorai-Raj, L. S. with help from.
(2015). \emph{MultcompView: Visualizations of paired comparisons}.
Retrieved from \url{https://CRAN.R-project.org/package=multcompView}

\hypertarget{ref-R-purrr}{}
Henry, L., \& Wickham, H. (2019). \emph{Purrr: Functional programming
tools}. Retrieved from \url{https://CRAN.R-project.org/package=purrr}

\hypertarget{ref-R-ggpubr}{}
Kassambara, A. (2019). \emph{Ggpubr: 'Ggplot2' based publication ready
plots}. Retrieved from \url{https://CRAN.R-project.org/package=ggpubr}

\hypertarget{ref-R-emmeans}{}
Lenth, R. (2019). \emph{Emmeans: Estimated marginal means, aka
least-squares means}. Retrieved from
\url{https://CRAN.R-project.org/package=emmeans}

\hypertarget{ref-R-lsmeans}{}
Lenth, R. V. (2016). Least-squares means: The R package lsmeans.
\emph{Journal of Statistical Software}, \emph{69}(1), 1--33.
doi:\href{https://doi.org/10.18637/jss.v069.i01}{10.18637/jss.v069.i01}

\hypertarget{ref-R-sjstats}{}
Lüdecke, D. (2019). \emph{Sjstats: Statistical functions for regression
models (version 0.17.5)}.
doi:\href{https://doi.org/10.5281/zenodo.1284472}{10.5281/zenodo.1284472}

\hypertarget{ref-R-lsr}{}
Navarro, D. (2015). \emph{Learning statistics with r: A tutorial for
psychology students and other beginners. (version 0.5)}. Adelaide,
Australia: University of Adelaide. Retrieved from
\url{http://ua.edu.au/ccs/teaching/lsr}

\hypertarget{ref-R-base}{}
R Core Team. (2018). \emph{R: A language and environment for statistical
computing}. Vienna, Austria: R Foundation for Statistical Computing.
Retrieved from \url{https://www.R-project.org/}

\hypertarget{ref-R-psych}{}
Revelle, W. (2018). \emph{Psych: Procedures for psychological,
psychometric, and personality research}. Evanston, Illinois:
Northwestern University. Retrieved from
\url{https://CRAN.R-project.org/package=psych}

\hypertarget{ref-R-fancycut}{}
Rich, A. (2018). \emph{Fancycut: A fancy version of 'base::cut'}.
Retrieved from \url{https://CRAN.R-project.org/package=fancycut}

\hypertarget{ref-R-numform}{}
Rinker, T. W. (2018). \emph{numform: A publication style number and plot
formatter}. Retrieved from \url{http://github.com/trinker/numform}

\hypertarget{ref-R-MASS}{}
Venables, W. N., \& Ripley, B. D. (2002). \emph{Modern applied
statistics with s} (Fourth.). New York: Springer. Retrieved from
\url{http://www.stats.ox.ac.uk/pub/MASS4}

\hypertarget{ref-R-reshape2}{}
Wickham, H. (2007). Reshaping data with the reshape package.
\emph{Journal of Statistical Software}, \emph{21}(12), 1--20. Retrieved
from \url{http://www.jstatsoft.org/v21/i12/}

\hypertarget{ref-R-ggplot2}{}
Wickham, H. (2016). \emph{Ggplot2: Elegant graphics for data analysis}.
Springer-Verlag New York. Retrieved from
\url{https://ggplot2.tidyverse.org}

\hypertarget{ref-R-haven}{}
Wickham, H., \& Miller, E. (2019). \emph{Haven: Import and export
'spss', 'stata' and 'sas' files}. Retrieved from
\url{https://CRAN.R-project.org/package=haven}

\hypertarget{ref-R-dplyr}{}
Wickham, H., Fran\textless{}U+00E7\textgreater{}ois, R., Henry, L., \&
Müller, K. (2019). \emph{Dplyr: A grammar of data manipulation}.
Retrieved from \url{https://CRAN.R-project.org/package=dplyr}

\hypertarget{ref-R-papaja}{}
Aust, F., \& Barth, M. (2018). \emph{papaja: Create APA manuscripts with
R Markdown}. Retrieved from \url{https://github.com/crsh/papaja}

\hypertarget{ref-R-magrittr}{}
Bache, S. M., \& Wickham, H. (2014). \emph{Magrittr: A forward-pipe
operator for r}. Retrieved from
\url{https://CRAN.R-project.org/package=magrittr}

\hypertarget{ref-R-pwr}{}
Champely, S. (2018). \emph{Pwr: Basic functions for power analysis}.
Retrieved from \url{https://CRAN.R-project.org/package=pwr}

\hypertarget{ref-R-car}{}
Fox, J., \& Weisberg, S. (2019). \emph{An R companion to applied
regression} (Third.). Thousand Oaks CA: Sage. Retrieved from
\url{https://socialsciences.mcmaster.ca/jfox/Books/Companion/}

\hypertarget{ref-R-carData}{}
Fox, J., Weisberg, S., \& Price, B. (2018). \emph{CarData: Companion to
applied regression data sets}. Retrieved from
\url{https://CRAN.R-project.org/package=carData}

\hypertarget{ref-R-multcompView}{}
Graves, S., Piepho, H.-P., \& Sundar Dorai-Raj, L. S. with help from.
(2015). \emph{MultcompView: Visualizations of paired comparisons}.
Retrieved from \url{https://CRAN.R-project.org/package=multcompView}

\hypertarget{ref-R-purrr}{}
Henry, L., \& Wickham, H. (2019). \emph{Purrr: Functional programming
tools}. Retrieved from \url{https://CRAN.R-project.org/package=purrr}

\hypertarget{ref-R-ggpubr}{}
Kassambara, A. (2019). \emph{Ggpubr: 'Ggplot2' based publication ready
plots}. Retrieved from \url{https://CRAN.R-project.org/package=ggpubr}

\hypertarget{ref-R-emmeans}{}
Lenth, R. (2019). \emph{Emmeans: Estimated marginal means, aka
least-squares means}. Retrieved from
\url{https://CRAN.R-project.org/package=emmeans}

\hypertarget{ref-R-lsmeans}{}
Lenth, R. V. (2016). Least-squares means: The R package lsmeans.
\emph{Journal of Statistical Software}, \emph{69}(1), 1--33.
doi:\href{https://doi.org/10.18637/jss.v069.i01}{10.18637/jss.v069.i01}

\hypertarget{ref-R-sjstats}{}
Lüdecke, D. (2019). \emph{Sjstats: Statistical functions for regression
models (version 0.17.5)}.
doi:\href{https://doi.org/10.5281/zenodo.1284472}{10.5281/zenodo.1284472}

\hypertarget{ref-R-lsr}{}
Navarro, D. (2015). \emph{Learning statistics with r: A tutorial for
psychology students and other beginners. (version 0.5)}. Adelaide,
Australia: University of Adelaide. Retrieved from
\url{http://ua.edu.au/ccs/teaching/lsr}

\hypertarget{ref-R-base}{}
R Core Team. (2018). \emph{R: A language and environment for statistical
computing}. Vienna, Austria: R Foundation for Statistical Computing.
Retrieved from \url{https://www.R-project.org/}

\hypertarget{ref-R-psych}{}
Revelle, W. (2018). \emph{Psych: Procedures for psychological,
psychometric, and personality research}. Evanston, Illinois:
Northwestern University. Retrieved from
\url{https://CRAN.R-project.org/package=psych}

\hypertarget{ref-R-fancycut}{}
Rich, A. (2018). \emph{Fancycut: A fancy version of 'base::cut'}.
Retrieved from \url{https://CRAN.R-project.org/package=fancycut}

\hypertarget{ref-R-numform}{}
Rinker, T. W. (2018). \emph{numform: A publication style number and plot
formatter}. Retrieved from \url{http://github.com/trinker/numform}

\hypertarget{ref-R-MASS}{}
Venables, W. N., \& Ripley, B. D. (2002). \emph{Modern applied
statistics with s} (Fourth.). New York: Springer. Retrieved from
\url{http://www.stats.ox.ac.uk/pub/MASS4}

\hypertarget{ref-R-reshape2}{}
Wickham, H. (2007). Reshaping data with the reshape package.
\emph{Journal of Statistical Software}, \emph{21}(12), 1--20. Retrieved
from \url{http://www.jstatsoft.org/v21/i12/}

\hypertarget{ref-R-ggplot2}{}
Wickham, H. (2016). \emph{Ggplot2: Elegant graphics for data analysis}.
Springer-Verlag New York. Retrieved from
\url{https://ggplot2.tidyverse.org}

\hypertarget{ref-R-haven}{}
Wickham, H., \& Miller, E. (2019). \emph{Haven: Import and export
'spss', 'stata' and 'sas' files}. Retrieved from
\url{https://CRAN.R-project.org/package=haven}

\hypertarget{ref-R-dplyr}{}
Wickham, H., Fran\textless{}U+00E7\textgreater{}ois, R., Henry, L., \&
Müller, K. (2019). \emph{Dplyr: A grammar of data manipulation}.
Retrieved from \url{https://CRAN.R-project.org/package=dplyr}

\endgroup

\end{document}
